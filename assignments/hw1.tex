\documentclass[letter]{article}
\usepackage[margin=1in]{geometry}
\usepackage{amsmath, amssymb}
\usepackage{enumerate}
\usepackage{fancyhdr}

\pagestyle{fancy}
\fancyhead[L]{STAT 542 HW1}
\fancyhead[R]{Xin Yin}

\begin{document}
    % Statistical Inference 1.6, 1.7, 1.12, 1.13, 1.18, 1.24, 1.25

    \section*{1.6}
    Define event $U$ = {first penny is head}, $W$ = {second penny is head}. 
    Because outcomes from two coins are independent, we have,
    \begin{align*}
    p_0 & = P(\text{0 heads occur}) = P(U^c \cap W^c) = P(U^c)P(W^c) = (1-u)(1-w) \\
    p_1 & = P(\text{1 head occurs}) = P((U \cap W^c) \cup (U^c \cap W)) = u(1-w) + w(1-u) \\
    p_2 & = P(\text{2 heads occur}) = P(U \cap W) = P(U)P(W) = uw
    \end{align*}

    If $p_0 = p_1 = p_2$, we can have, $uw = 1/3, u + w = 1$, or,
    \[ u(1-u) = \frac{1}{3} \Rightarrow u^2 - u + \frac{1}{3} = 0 \]
    For this quadratic equation, $\Delta < 0$. Therefore, no real $u$ and $w$ can be chosen such that $p_0 = p_1 = p_2$.

    \section*{1.7}
    \begin{enumerate}[(a)]
        \item Based on the fact that the probability of hitting a region is
        proportional to area, we can define probability function
        $P(\text{scoring $i$ points}), i=0,\dots,5$ as,

        \[
        P(\text{scoring}~i~\text{points}) = \frac{\text{Area of
        region}~i}{\text{Area of wall}} = 
        \begin{cases}
        \frac{\pi ((6-i)r/5)^2 - \pi((5-i)r/5)^2}{A} = \frac{\pi r^2
        \left((6-i)^2 - (5-i)^2\right)}{25A} &,  i = 1, \dots, 5\\
        \frac{A-\pi r^2}{A} &, i = 0
        \end{cases}
        \]
        \item The conditional probability 
        $P(\text{scoring $i$ points}|\text{board is hit})$, using the
        definition of conditional probability is,
        \[
        P(\text{scoring $i$ points}|\text{board is hit}) = \frac{
        P(\text{scoring $i$ points} \cap \text{board is hit})}{
        P(\text{board is hit})
        }.
        \]

        Note that event \emph{board is hit} implies player scores at least $1$
        point, \emph{i.e.} $i > 1$. 
        Thus $P(\text{scoring $i$ points} \cap \text{board is hit})$ is the
        probability defined in (a) under condition of $i=1, \dots 5$, which is,
        \[
        P(\text{scoring $i$ points} \cap \text{board is hit}) = \frac{\pi r^2
        \left((6-i)^2 - (5-i)^2\right)}{5^2A}.
        \]
        And $P(\text{board is hit}$) is nothing but, 
        \[
        P(\text{board is hit}) = \frac{\pi r^2}{A}.
        \]

        Therefore, we have,
        \[
        P(\text{scoring $i$ points}|\text{board is hit}) = \frac{(6-i)^2 -
        (5-i)^2}{5^2},
        \]
        which is exactly the probability in Example 1.2.7.

    \end{enumerate}

    \section*{1.12}
    \begin{enumerate}[(a)]
    \item Let $A \in \mathcal{B}, B \in \mathcal{B}$ be two disjoint sets, using Axiom of Countable
    Additivity, we have,
    \[
    P(A \cup B \cup \emptyset \cup \emptyset \cup \dots) = P(A) + P(B) + P(\emptyset) + \dots = P(A \cup B) = P(A) + P(B)
    \]
    \item Let $A_1, A_2, \dots$ be pairwise disjoint sets, $A = \bigcup_{i=1}^{\infty} A_i$. And let 
    \[
    B_1 = A_1, B_2 = A_1 \cup A_2, \dots, B_n = A_1 \cup \dots \cup A_n, \dots
    \], we have that,
    \[
    B_1 \subset B_2 \subset \dots \subset B_n \subset \dots \uparrow A
    \]
    which implies,
    \[
    A \backslash B_1 \supset A \backslash B_2 \supset \dots \supset A
    \backslash B_n \supset \dots \downarrow \emptyset
    \]

    Using Axiom of Finite Additivity, we have, for two disjoint sets $B_n$ and
    $A \backslash B_n$ in $A$,
    \[
    P(A) = P\left(B_n \cup A \backslash B_n\right) = P(B_n) + P(A \backslash
    B_n) = P(\bigcup_{i=1}^n A_i) + P(A \backslash B_n)
    \]

    As $n \to \infty$, we have,
    \[
    P(A) = \lim_{n \to \infty} P(\bigcup_{i=1}^n A_i) + \lim_{n \to \infty} P(A
    \backslash B_n)
    \]

    Using Axiom of Continuity, we know that $\lim_{n \to \infty} P(A \backslash B_n) = 0$, therefore,
    \[
    P(A) = P(\bigcup_{i=1}^{\infty} A_i) = \lim_{n \to \infty} P(\bigcup_{i=1}^n A_i) = \lim_{n \to \infty} \sum_{i=1}^n P(A_i) 
    = \sum_{i=1}^{\infty} P(A_i)
    \]

    \end{enumerate}
    \section*{1.13}
    If $P(A) = 1/3, P(B^c) = 1/4$, we know that $P(B) = 1 - P(B^c) = 3/4$. 
    Suppose $A$ and $B$ are disjoint, we know that, $\{A, B, (A^c \cap B^c)\}$ is a partition of $S$.
    Then, we have $P\left(A \cup B \cup (A^c \cap B^c) \right) = P(A) + P(B) + P(A^c \cap B^c) = 1$.

    However, $P(A) + P(B) = \frac{1}{3} + \frac{3}{4} = \frac{13}{12} > 1, P(A^c \cap B^c) \geq 0$. 
    Therefore, $A$ and $B$ can't be disjoint.

    \section*{1.18}
    To place $n$ balls into $n$ cells, if there's exactly only one cell is empty, it suggests that $n-2$ cells have
    exactly one ball and only $1$ cell has two balls. 
    
    \begin{enumerate}[(a)]
    \item If the balls are disinct, \emph{i.e.} this is an ordered experiment, the sample space $\Omega$ 
    will consist of $n^n$ outcomes of different placement. Among the $n^n$ outcomes, we now have to 
    discuss how many of them satisfy the requirement of only one empty cell.

    We have $n!$ ways to place $n$ balls into $n$ cells such that each cell has exactly one ball.
    Next, we randomly pick $1$ out of $n$ cell, grab the ball from the cell and put it into another cell,
    which can be chosen from the remaining $n-1$ cells. The order of two balls does not matter, so 
    in total we have $\frac{n (n-1)}{2}$ ways to do this.
    Overall, we have $\frac{n(n-1) n!}{2} = \binom{n}{2} n!$ ways to have exactly one empty cell.

    The probability is therefore, $\frac{\binom{n}{2} n!}{n^n}$.

    \item If the balls are indistinguishable, the sample space $\Omega$ will only contain 
    $\binom{n+n-1}{n}$ outcomes of different placement. 
    
    Since all balls are the same, we have exactly $1$ way to place one ball in each cell. 
    Now we need to randomly choose two cells, grab ball from cell A and put it in B. We have $n$ ways
    to choose A and $n-1$ ways to choose B. In total, $n(n-1)$ ways to have exactly one empty cell.

    The probability under unordered scenario is, $\frac{n(n-1)}{\binom{n+1-1}{n}}$.

    \end{enumerate}

    \section*{1.24}
    Define a series of events $E_i$, where $E_i$ = {head first appears on $i$-th toss. For a generic probability $P(\text{head})=p$ having a head in each toss, we have, $P(E_i) = (1-p)^{i-1} \times p, \text{for} i = 1,2,\dots$.
    
    Because two players toss coin alternately, define event $A$ = {A wins the game}, which can be written as, $A = E_1 \cup E_3 \cup E_5 \cup \dots = \bigcup_{j=1}^{\infty} E_{2j-1}$.
    \begin{enumerate}[(a)]
        \item A fair coin implies $p = 0.5$. Then we have,
        \[ 
        P(A) = P(\bigcup_{j=1}^{\infty} E_{2j-1}) = \sum_{j=1}^{\infty} P(E_{2j-1}) = 1/2 + 1/8 + 1/32 + ... = 2/3 \approx 0.667 
        \]
        \item For a more generic $p$, we have,
       
        \begin{align*}
        P(A) & = P(\bigcup_{j=1}^{\infty} E_{2j-1}) = \sum_{j=1}^{\infty} P(E_{2j-1})\\
        & = p + (1-p)^2 p + (1-p)^4 p + \dots = p\left(1 + (1-p)^2 + (1-p)^4 + \dots\right) \\
        & = p \sum_{n=0}^{\infty} (1-p)^{2n} = p \left(\sum_{n=1}^{\infty} (1-p)^n - \sum_{n=1}^{\infty} (1-p)^{2n-1}\right)\\
        & = p \left(\sum_{n=1}^{\infty} (1-p)^n - (1-p) \sum_{n=0}^{\infty} (1-p)^{2n} \right) \\
        & = p \left(\frac{1}{1-(1-p)} - \frac{1-p}{1-(1-p)^2}\right) = \frac{1}{2-p}
        \end{align*}
        
        \item From (b), we already know that, $P(A) = \frac{1}{2-p}$. 
        For all $0 < p < 1, \frac{1}{2-p} > \frac{1}{2-0} = \frac{1}{2}$. Therefore, $P(\text{A wins}) > \frac{1}{2}$.
    \end{enumerate}
    
    \section*{1.25}
    For the children for the Smiths, the sample space of their sex $\Omega$ is $\{GG, GB, BG, BB\}$.
    If we condition on the knowledge that at least of these two children is a boy, our sample space
    is reduced to $\{GB, BG, BB\}$.

    Therefore, $P(\text{both children are boys}|\text{at least one is boy}) = \frac{1}{3}$.
\end{document}
